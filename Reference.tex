\documentclass[hidelinks,12pt]{article}
\usepackage[utf8]{inputenc}
\usepackage{graphicx} % Allows you to insert figures
\usepackage{amsmath,amssymb,amsthm,mathtools,empheq,tensor,braket} % Allows you to do equations
\usepackage{fancyhdr} % Formats the header
\usepackage{geometry} % Formats the paper size, orientation, and margins
\usepackage{hyperref} % Allows linking sections to each other

\setlength{\parindent}{0pt} % No paragraph indents
\setlength{\parskip}{1em} % Paragraphs separated by one line
\renewcommand{\headrulewidth}{0pt} % Removes line in header
\geometry{legalpaper, portrait, margin=1in}
\setlength{\headheight}{14.49998pt}
\newcommand{\pagesetup}{\pagestyle{fancy}\fancyhf{}\rhead{\thepage}\lhead{Stuyvesant Class of 2022; Complex Analysis Reference Sheet}}

\theoremstyle{definition}
\newtheorem{definition}{Definition}[section]
\newtheorem{theorem}{Theorem}[section]

%Operations for complex numbers
\newcommand{\modulus}[1]{\left|#1\right|} % Modulus of a complex number
\newcommand{\conj}[1]{\overline{#1}} % Conjugate of a complex number
\newcommand{\nhd}[1]{Nbhd{#1}} % Set of neighborhoods of a point
\newcommand{\puncnhd}[1]{Nbhd^*{#1}} % Set of punctured neighborhoods of a point

\begin{document}
  \begin{titlepage}
     \begin{center}
          \vspace*{5cm}

          \Huge{Complex Analysis Reference Sheet}


          \vspace{3 cm}
          \Large{Stuyvesant Class of 2022}

          \vspace{0.25cm}
          \large{Nate Strout, Joshua Yagupsky, Francis Zweifler}

          \vspace{3 cm}
          \Large{Created February 15th, 2022}

          \vspace{0.25 cm}
          Complex Calculus, Mr. Stern
         \vfill
      \end{center}
  \end{titlepage}

  \thispagestyle{empty}
  \tableofcontents
  \pagebreak

  \pagesetup
  \section{THE COMPLEX NUMBER SYSTEM}
  \subsection{The Algebra of Complex Numbers}
  \subsection{The Geometry of Complex Numbers}
  \subsubsection{Möbius Transformations and the Riemann Sphere}
  \section{COMPLEX FUNCTIONS}
  \subsection{The Complex Exponential}
  \section{TOPOLOGY OF $\mathbb{C}$}
  \subsection{Compact Sets}
  \subsection{Sequences in $\mathbb{C}$}
  \subsection{Limits of functions, continuity}
  \begin{definition}[Limit of a function, $(\epsilon, \delta)$ definition]
    \label{def:edlim}
    A function $f: D \to \mathbb{C}$ is said to have a \emph{limit} of $L$ as $z \to a$ if $\forall\epsilon > 0, \exists\delta > 0, \forall z \in D: (0 < \modulus{z-a} < \delta \implies \modulus{f(z)-L} < \epsilon)$.
  \end{definition}
\end{document}
