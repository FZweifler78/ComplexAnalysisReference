\documentclass[hidelinks,12pt]{article}
\usepackage[utf8]{inputenc}
\usepackage{graphicx} % Allows you to insert figures
\usepackage{amsmath,amssymb,amsthm,mathtools,empheq,tensor,braket} % Allows you to do equations
\usepackage{fancyhdr} % Formats the header
\usepackage{geometry} % Formats the paper size, orientation, and margins
\usepackage{hyperref} % Allows linking sections to each other

\setlength{\parindent}{0pt} % No paragraph indents
\setlength{\parskip}{1em} % Paragraphs separated by one line
\renewcommand{\headrulewidth}{0pt} % Removes line in header
\geometry{legalpaper, portrait, margin=1in}
\setlength{\headheight}{14.49998pt}
\newcommand{\pagesetup}{\pagestyle{fancy}\fancyhf{}\rhead{\thepage}\lhead{Stuyvesant Class of 2022: Complex Analysis Reference Sheet}}
\addtocontents{toc}{\protect\thispagestyle{empty}}

\theoremstyle{definition}
\newtheorem{definition}{Definition}[section]
\newtheorem{theorem}{Theorem}[section]
\newtheorem{corollary}{Corollary}[theorem]

%Operations for complex numbers
\newcommand{\modulus}[1]{\left|#1\right|} % Modulus of a complex number
\newcommand{\conj}[1]{\overline{#1}} % Conjugate of a complex number

\begin{document}
  \begin{titlepage}
     \begin{center}
          \vspace*{5cm}

          \Huge{Complex Analysis Reference Sheet}


          \vspace{3 cm}
          \Large{Stuyvesant Class of 2022}

          \vspace{0.25cm}
          \large{Nathaniel J. Strout, Joshua L. Yagupsky, Francis Zweifler}

          \vspace{3 cm}
          \Large{Created February 15th, 2022}

          \vspace{0.25 cm}
          Complex Calculus, Mr. Stern
         \vfill
      \end{center}
  \end{titlepage}

  \pagestyle{empty}
  \tableofcontents
  \cleardoublepage

  \pagesetup
  \section{THE COMPLEX NUMBER SYSTEM}
  \subsection{The Algebra of Complex Numbers}
    \begin{definition} [Complex Numbers]
      \label{def:cmpnum}
      The \emph{complex numbers} are the set
      \begin{equation*}
        \mathbb{C} := \{\,[(a,b)] \mid a,b \in \mathbb{R}\,\}
      \end{equation*}
      where
      \begin{equation*}
        [(a,b)] :=
        \begin{cases}
          a & b = 0 \\
          (a,b) & b \neq 0
        \end{cases}
      \end{equation*}
    \end{definition}
    \begin{definition} [Real and Imaginary Parts of an element of $\mathbb{C}$]
      The real part of $z \in \mathbb{C}$ where $z=[(a,b)]$ is defined as:
      \begin{equation*}
        Re(z) := a.
      \end{equation*}
      Similarly, the imaginary part of $z \in \mathbb{C}$ is defined by $Im(z):=b$.
    \end{definition}
    \begin{definition} [Addition and Multiplication in $\mathbb{C}$]
      For $[(a,b)],[(c,d)]\in \mathbb{C}$, $[(a,b)]+[(c,d)]:=[(a+c,b+d)]$. Additionally, we will define multiplication as follows:
      \begin{equation*}
        [(a,b)]\cdot[(c,d)] := [(ac - bd, ad + bc)].
      \end{equation*}
      This may seem to be an arbitrary defintion of multiplication, but as we explore the properties of this definition its motivation will become apparant.
    \end{definition}
    \begin{theorem} [Properties of Addition and Multiplication $\mathbb{C}$]
      We can now verify a number of properties that will allow this rigorous bracketed point construction of $\mathbb{C}$ to be recgonized as equivalent to the familiar (non-rigorous) introduction of $\mathbb{C}$.
      Starting with properties of addition in $\mathbb{C}$, for $[(a,b)],[(c,d)] \in \mathbb{C}$ we have commutativity, associativity, and an identity.

      Starting with commutativity, we know $[(a,b)]+[(c,d)]=[(a+c,b+d)]=[(c+a,d+b)]$ by the commutativity of $\mathbb{R}$. $[(c,d)]+[(a,b)]=[(c+a,d+b)]$, and thus we have $[(a,b)]+[(c,d)]=[(c,d)]+[(a,b)]$.

      Next, associativity for $[(a,b)],[(c,d)],[(e,f)]\in \mathbb{C}$ is proven easily as well: We know $([(a,b)]+[(c,d)])+[(e,f)]=[((a+c)+e,(b+d)+f)]$. By the associativity of $\mathbb{R}$, we have $[(a+(c+e),b+(d+f))]=[(a,b)]+([(c,d)]+[(e,f)])$, and thus addition in $\mathbb{C}$ is associative.

      Finally we have an identity element, $[(0,0)]$. Note that $[(0,0)]$ is equal to the real number $0$. We quickly verify that it is indeed an indentity: For any $[(a,b)]\in \mathbb{C}$, $[(a,b)]+[(0,0)]=[(a+0,b+0)]=[(a,b)]$.

      Next, we verify the same properties for multiplication. (In these more or less straightforward proofs, it will become clear how notationally clumsy our bracketed point notation is, and we will abadon it soon after these proofs. Terrence Tao calls this type of idea mathematical scaffolding in his book Analysis I: necessary to sturdy and rigorous construction, but discarded after use.)

      Once again, starting with commutativity, we know $[(a,b)]\cdot[(c,d)]=[(ac-bd,ad+bc)]$ by definition. Similarly, $[(c,d)]\cdot[(a,b)]=[(ca-db,cb+da)]$ and that is equal to $[(ac-bd,ad+bc)]$ by commutativity of multiplication and addtion in $\mathbb{R}$.

      Next we prove associativity. We know $([(a,b)]\cdot[(c,d)])\cdot[(e,f)]=([(ac-bd,ad+bc)])\cdot[(e,f)]=[((ac-bd)e-(ad+bc)f,(ac-bd)f+(ad+bc)e]$. This is equal to $[(a,b)]\cdot([(c,d)]\cdot[(e,f)])$ by the standard properties of multiplication and addition in $\mathbb{R}$.

      As an intermediate step, we can also use this oppurtunity to prove distributivity of multiplication over addition. For $[(a,b)],[(c,d)],[(e,f)]\in \mathbb{C}$ we know
      \begin{equation*}
        [(a,b)]\cdot([(c,d)]+[(e,f)])=[(a,b)]\cdot[(c+e,d+f)]=[(ac+ae-bd-bf,ad+af+bc+be)]
      \end{equation*}
      Then, by the properties of addition and multiplication in $\mathbb{R}$ we knnow
      \begin{equation*}
        [(ac-bd,ad+bc)]+[(ae-bf,af+be)]=[(a,b)]\cdot[(c,d)]+[(a,b)]\cdot[(e,f)]
      \end{equation*}
    \end{theorem} [Properties of Addition and Multiplication $\mathbb{C}$]








  \subsection{The Geometry of Complex Numbers}
  \subsubsection{Möbius Transformations and the Riemann Sphere}


  \section{COMPLEX FUNCTIONS}
  \subsection{The Complex Exponential}
  \subsection{Complex Trigonometry}
  \subsection{The Argument Functions and Complex Logarithm}


  \section{TOPOLOGY OF $\mathbb{C}$}
  \subsection{Neighborhoods, Open and Closed Sets}
  \subsubsection{Accumulation Points and the Closure of a Set}
  \subsection{Connectedness and Compactness}
  \subsection{Sequences in $\mathbb{C}$, Limits of Sequences}
  \subsection{Limits of Functions, Continuity}
  \subsubsection{Continuous Images of Connected and Compact Sets}


  \section{COMPLEX DIFFERENTIATION}
  \subsection{Differentiability and Analyticity}
  \subsubsection{The Cauchy-Riemann Equations}
  \subsection{Rules for Derivatives}
  \subsection{Conformal Maps}


  \section{COMPLEX INTEGRATION}
  \subsection{Real Integrals of Complex Functions}
  \subsection{Contours and Contour Integration}
  \subsubsection{Contour Parameterization}
  \subsection{Introduction to Cauchy's Theorem}
  \subsubsection{Tile-Centered Regions and the Not-Too-Bad Condition}
  \subsection{Cauchy's Integral Formula}

  \section{PROPERTIES OF ANALYTIC FUNCTIONS}
  \subsection{Taylor Series}
  \subsection{Meromorphic Functions, Classification of Singularities}
  \subsubsection{Properties of Essential Singularities}
\end{document}
