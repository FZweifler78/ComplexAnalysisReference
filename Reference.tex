\documentclass[hidelinks,12pt]{article}
\usepackage[utf8]{inputenc}
\usepackage{graphicx} % Allows you to insert figures
\usepackage{amsmath,amssymb,amsthm,mathtools,empheq,tensor,braket} % Allows you to do equations
\usepackage{fancyhdr} % Formats the header
\usepackage{geometry} % Formats the paper size, orientation, and margins
\usepackage{hyperref} % Allows linking sections to each other

\setlength{\parindent}{0pt} % No paragraph indents
\setlength{\parskip}{1em} % Paragraphs separated by one line
\renewcommand{\headrulewidth}{0pt} % Removes line in header
\geometry{legalpaper, portrait, margin=1in}
\setlength{\headheight}{14.49998pt}
\newcommand{\pagesetup}{\pagestyle{fancy}\fancyhf{}\rhead{\thepage}\lhead{Stuyvesant Class of 2022: Complex Analysis Reference Sheet}}
\addtocontents{toc}{\protect\thispagestyle{empty}}

\theoremstyle{definition}
\newtheorem{definition}{Definition}[section]
\newtheorem{theorem}{Theorem}[section]
\newtheorem{corollary}{Corollary}[theorem]

%Operations for complex numbers
\newcommand{\modulus}[1]{\left|#1\right|} % Modulus of a complex number
\newcommand{\conj}[1]{\overline{#1}} % Conjugate of a complex number

\begin{document}
  \begin{titlepage}
     \begin{center}
          \vspace*{5cm}

          \Huge{Complex Analysis Reference Sheet}


          \vspace{3 cm}
          \Large{Stuyvesant Class of 2022}

          \vspace{0.25cm}
          \large{Nathaniel J. Strout, Joshua L. Yagupsky, Francis Zweifler}

          \vspace{3 cm}
          \Large{Created February 15th, 2022}

          \vspace{0.25 cm}
          Complex Calculus, Mr. Stern
         \vfill
      \end{center}
  \end{titlepage}

  \pagestyle{empty}
  \tableofcontents
  \cleardoublepage

  \pagesetup
  \section{THE COMPLEX NUMBER SYSTEM}
  \subsection{The Algebra of Complex Numbers}
    \begin{definition} [Complex Numbers]
      \label{def:cmpnum}
      The \emph{complex numbers} are the set
      \begin{equation*}
        \mathbb{C} := \{\,[(a,b)] \mid a,b \in \mathbb{R}\,\}
      \end{equation*}
      where
      \begin{equation*}
        [(a,b)] :=
        \begin{cases}
          a & b = 0 \\
          (a,b) & b \neq 0
        \end{cases}
      \end{equation*}
    \end{definition}
    \begin{definition} [Adding Complex Numbers]
      
    \end{definition}
  \subsection{The Geometry of Complex Numbers}
  \subsubsection{Möbius Transformations and the Riemann Sphere}


  \section{COMPLEX FUNCTIONS}
  \subsection{The Complex Exponential}
  \subsection{Complex Trigonometry}
  \subsection{The Argument Functions and Complex Logarithm}


  \section{TOPOLOGY OF $\mathbb{C}$}
  \subsection{Neighborhoods, Open and Closed Sets}
  \subsubsection{Accumulation Points and the Closure of a Set}
  \subsection{Connectedness and Compactness}
  \subsection{Sequences in $\mathbb{C}$, Limits of Sequences}
  \subsection{Limits of Functions, Continuity}
  \subsubsection{Continuous Images of Connected and Compact Sets}


  \section{COMPLEX DIFFERENTIATION}
  \subsection{Differentiability and Analyticity}
  \subsubsection{The Cauchy-Riemann Equations}
  \subsection{Rules for Derivatives}
  \subsection{Conformal Maps}


  \section{COMPLEX INTEGRATION}
  \subsection{Real Integrals of Complex Functions}
  \subsection{Contours and Contour Integration}
  \subsubsection{Contour Parameterization}
  \subsection{Introduction to Cauchy's Theorem}
  \subsubsection{Tile-Centered Regions and the "Not-Too-Bad" Condition}
  \subsection{Cauchy's Integral Formula}

  \section{PROPERTIES OF ANALYTIC FUNCTIONS}
  \subsection{Taylor Series}
  \subsection{Meromorphic Functions, Classification of Singularities}
  \subsubsection{Properties of Essential Singularities}
\end{document}
