\documentclass{refbook}
\begin{document}
\section{THE COMPLEX NUMBER SYSTEM}
\subsection{The Algebra of Complex Numbers}
\begin{definition} [Complex Numbers]
\label{def:cmpnum}
The \emph{complex numbers} are the set
\begin{equation*}
\mathbb{C} := \{\,[(a,b)] \mid a,b \in \mathbb{R}\,\}
\end{equation*}
where
\begin{equation*}
[(a,b)] :=
\begin{cases}
a & b = 0 \\
(a,b) & b \neq 0
\end{cases}
\end{equation*}
\end{definition}
\begin{definition} [Real and Imaginary Parts of an element of $\mathbb{C}$]
The real part of $z \in \mathbb{C}$ where $z=[(a,b)]$ is defined as:
\begin{equation*}
Re(z) := a.
\end{equation*}
Similarly, the imaginary part of $z \in \mathbb{C}$ is defined by $Im(z):=b$.
\end{definition}
\begin{definition} [Addition and Multiplication in $\mathbb{C}$]
For $[(a,b)],[(c,d)]\in \mathbb{C}$, $[(a,b)]+[(c,d)]:=[(a+c,b+d)]$. Additionally, we will define multiplication as follows:
\begin{equation*}
[(a,b)]\cdot[(c,d)] := [(ac - bd, ad + bc)].
\end{equation*}
This may seem to be an arbitrary defintion of multiplication, but as we explore the properties of this definition its motivation will become apparant.
\end{definition}
\begin{theorem} [Properties of Addition and Multiplication $\mathbb{C}$]
We can now verify a number of properties that will allow this rigorous bracketed point construction of $\mathbb{C}$ to be recgonized as equivalent to the familiar (non-rigorous) introduction of $\mathbb{C}$.
Starting with properties of addition in $\mathbb{C}$, for $[(a,b)],[(c,d)] \in \mathbb{C}$ we have commutativity, associativity, and an identity.

Starting with commutativity, we know $[(a,b)]+[(c,d)]=[(a+c,b+d)]=[(c+a,d+b)]$ by the commutativity of $\mathbb{R}$. $[(c,d)]+[(a,b)]=[(c+a,d+b)]$, and thus we have $[(a,b)]+[(c,d)]=[(c,d)]+[(a,b)]$.

Next, associativity for $[(a,b)],[(c,d)],[(e,f)]\in \mathbb{C}$ is proven easily as well: We know $([(a,b)]+[(c,d)])+[(e,f)]=[((a+c)+e,(b+d)+f)]$. By the associativity of $\mathbb{R}$, we have $[(a+(c+e),b+(d+f))]=[(a,b)]+([(c,d)]+[(e,f)])$, and thus addition in $\mathbb{C}$ is associative.

Finally we have an identity element, $[(0,0)]$. Note that $[(0,0)]$ is equal to the real number $0$. We quickly verify that it is indeed an indentity: For any $[(a,b)]\in \mathbb{C}$, $[(a,b)]+[(0,0)]=[(a+0,b+0)]=[(a,b)]$.

Next, we verify the same properties for multiplication. (In these more or less straightforward proofs, it will become clear how notationally clumsy our bracketed point notation is, and we will abadon it soon after these proofs. Terrence Tao calls this type of idea mathematical scaffolding in his book Analysis I: necessary to sturdy and rigorous construction, but discarded after use.)

Once again, starting with commutativity, we know $[(a,b)]\cdot[(c,d)]=[(ac-bd,ad+bc)]$ by definition. Similarly, $[(c,d)]\cdot[(a,b)]=[(ca-db,cb+da)]$ and that is equal to $[(ac-bd,ad+bc)]$ by commutativity of multiplication and addtion in $\mathbb{R}$.

Next we prove associativity. We know $([(a,b)]\cdot[(c,d)])\cdot[(e,f)]=([(ac-bd,ad+bc)])\cdot[(e,f)]=[((ac-bd)e-(ad+bc)f,(ac-bd)f+(ad+bc)e]$. This is equal to $[(a,b)]\cdot([(c,d)]\cdot[(e,f)])$ by the standard properties of multiplication and addition in $\mathbb{R}$.

As an intermediate step, we can also use this oppurtunity to prove distributivity of multiplication over addition. For $[(a,b)],[(c,d)],[(e,f)]\in \mathbb{C}$ we know
\begin{equation*}
[(a,b)]\cdot([(c,d)]+[(e,f)])=[(a,b)]\cdot[(c+e,d+f)]=[(ac+ae-bd-bf,ad+af+bc+be)]
\end{equation*}
Then, by the properties of addition and multiplication in $\mathbb{R}$ we knnow
\begin{equation*}
[(ac-bd,ad+bc)]+[(ae-bf,af+be)]=[(a,b)]\cdot[(c,d)]+[(a,b)]\cdot[(e,f)]
\end{equation*}
\end{theorem} 








\subsection{The Geometry of Complex Numbers}
\subsubsection{Möbius Transformations and the Riemann Sphere}


\section{COMPLEX FUNCTIONS}
\subsection{The Complex Exponential}
\subsection{Complex Trigonometry}
\subsection{The Argument Functions and Complex Logarithm}


\section{TOPOLOGY OF THE COMPLEX PLANE}
\subsection{Neighborhoods, Open and Closed Sets}
\begin{definition}[Open Disk]
The \emph{open disk} of radius $r$ around the point $p$ is defined as follows:
\begin{equation*}\disk{r}{p} = \{z \in \mathbb{C} \mid \modulus{z-p} < r\}\end{equation*}
\end{definition}
\begin{definition}[Neighborhood of a Point]
A set $S \subseteq \mathbb{C}$ is called a \emph{neighborhood} of a point $p$ if there exists some $r > 0$ such that $\disk{p}{r} \subseteq S$.
We write this as $S \in \nbhd{p}$, where $\nbhd{p}$ denotes the \emph{neighborhood-system} of $p$.
\end{definition}
\begin{definition}[Punctured Disk]
The \emph{punctured disk} of radius $r$ around the point $p$ is defined as follows:
\begin{equation*}\pdisk{r}{p} = \{z \in \mathbb{C} \mid 0<\modulus{z-p} < r\}\end{equation*}
\end{definition}
\begin{definition}[Punctured Neighborhood of a Point]
A set $S \subseteq \mathbb{C}$ is called a \emph{punctured neighborhood} of a point $p$ if there exists some $r > 0$ such that $\pdisk{p}{r} \subseteq S$ and $p \notin S$.
We write this as $S \in \pnbhd{p}$, where $\pnbhd{p}$ denotes the \emph{punctured neighborhood-system} of $p$.
\end{definition}
\begin{definition}[Open Set]
A set $S$ is called an \emph{open set} iff $\forall z \in S, S \in \nbhd{z}$.
\end{definition}
\begin{lemma}[Open Disks are Open Sets]
Any open disk $\disk{p}{r}$ is an open set.
\end{lemma}
\begin{proof}
We need to show that $\disk{p}{r}$ is a neighborhood of everyone one of its points. Let $q \in \disk{p}{r}$. We can show that $\disk{q}{r-\modulus{p-q}} \subseteq \disk{p}{r}$. To do this, pick any point $z$ in the new disk. To show it's in our original disk, we need $\modulus{z-p} < r$. Since $z$ is in the second disk, we can write $\modulus{z-q} < r-\modulus{p-q}$. We can then use the triangle inequality to show that $\modulus{z-p} = \modulus{z-q+q-p} < \modulus{z-q}+\modulus{p-q} < r-\modulus{p-q}+\modulus{p-q} < r$. Therefore, our original open disk contains a smaller open disk around every point, making it a neighborhood of every one of its points, i.e. an open set.
\end{proof}
\begin{lemma}[Supersets of Neighborhoods]
If $N \in \nbhd{p}$ and $N \subseteq M$, $M \in \nbhd{p}$.
\end{lemma}
\begin{proof}
If $N \in \nbhd{p}$, then there exists some $\disk{p}{r} \subseteq N$. Since $N \subseteq M$, $\disk{p}{r} \subseteq M$. Therefore, $M \in \nbhd{p}$.
\end{proof}
\begin{lemma}[Properties of Open Sets]
The union of any collection of open sets is open, and the intersection of any \emph{finite} collection of open sets is open.
\end{lemma}
\begin{proof}
The union of any collection of sets is the set containing all of the points in all of the sets. Denote the union as $U$, and let $p$ be an element of one of the sets, $S$. Since $S$ is an open set, it is a neighborhood of $p$, and therefore there exists a $\disk{p}{r} \in S$.
Since $S$ is a subset of $U$, so is the disk, and since this is true of any point in the union, the union must be open.
If a point $p$ is in the intersection of a collection of sets $C$, it is in everyone one of the elements of the collection. Since each set in the collection is open, they are all neighborhoods of $p$. For every set $S$ in the collection, there is therefore some radius $r_S$ such that $\disk{p}{r_S} \in S$. Note that we can choose any radius less than or equal to $r_S$ and still get an open disk around $p$ fully contained in $S$
Since there are only finitely many sets in the collection, we can let $\epsilon = \min r_S$, which will be greater than $0$. But since $\epsilon \leq r_S$ for any $S$, $\disk{p}{\epsilon}$ will be a subset of every $S$, and therefore will be in the intersection. The intersection is therefore a neighborhood of all of its points, making it an open set. 
\end{proof}
\begin{definition}[Closed Set]
A set $S \subseteq \mathbb{C}$ is called \emph{closed} iff its complement $S^C$ is open.
\end{definition}
\begin{lemma}[Properties of Closed Sets]
The intersection of any collection of closed sets is closed, and the union of any \emph{finite} collection of closed sets is closed.
\end{lemma}
\begin{proof}
This proof follows directly from De Morgan's Laws for sets and \hyperlink{Properties of Open Sets}{the union and intersection properties of open sets}.
\end{proof}
\begin{definition}[Topology on a Set]
Given any set $X$, a \emph{topology} on $X$ $\tau_X$ is a collection of subsets called open sets, which satisfy the following properties:
\begin{itemize}
\item $\emptyset \in \tau_X, X \in \tau_X$
\item The union of a collection of elements in $\tau_X$ is in $\tau_X$
\item The intersection of a \emph{finite} collection of elements in $\tau_X$ is in $\tau_X$
\end{itemize}
\end{definition}
\begin{definition}[Relative Topology]
Given any type of set in the topology of $\mathbb{C}$ (open set, closed set, open disk, punctured disk, etc.) and some fixed subset $X \subseteq \mathbb{C}$, we can define a type of set \emph{relative} to $X$ as the intersection of that type of set with $X$. For instance, if $N \in \nbhd{p}$ and $p \in X$, then the intersection $N \cap X$ is called a \emph{relative neighborhood} of $p$, denoted as $N \cap X \in \rnbhd{p}{X}$. The collection of all open sets relative to $X$ forms a topology, called the \emph{relative topology} of $X$.
\end{definition}
\subsection{Accumulation Points and the Closure of a Set}
\begin{definition}[Accumulation Point]
A point $p$ is called an \emph{accumulation point} or \emph{limit point} of a set $S$ iff there does not exist a neigborhood $N \in \nbhd{p}$ such that $N \cap S = \emptyset$.
\end{definition}
\begin{lemma}[Accumulation Points of a Closed Set]
$K \subseteq \mathbb{C}$ is closed iff it contains all of its accumulation points.
\end{lemma}
\begin{proof}
First, we will prove that a closed set contains all of its limit points.
Suppose for the sake of contradiction that there exists a $p$ which is an accumulation point of $K$ and is in $K^C$. Since $K$ is closed, its complement is open, and therefore is a neighborhood of every one of its points. $p \in K^C$, so $K^C \in \nbhd{p}$. Therefore, there exists some disk centered at $p$ which lies entirely in $K^C$. But this disk is a neighborhood of $p$ which does not contain any points in $K$, which means $p$ cannot be an accumulation point of $S$.
Now for the converse. Suppose $K$ contains all of its limit points. This means that there isn't a limit point of $K$ in $K^C$. Thus, for any point $p \in K^C$, there exists a neighborhood of $p$ which is entirely in $K^C$. But since \hyperlink{Supersets of Neighborhoods}{any superset of a neighborhood of $p$ is still a neighborhood of $p$}, $K^C \in \nbhd{p}$. $K^C$ is therefore a neighborhood of all of its points, meaning it is open, and $K$ is closed.
\end{proof}
\begin{definition}[Closure of a Set]
Given a set $S$, its \emph{closure} $\overline{S}$ is the union of $S$ and all of its accumulation points.
\end{definition}
\begin{lemma}[Closure of a Closed Set]
$K$ is a closed set iff $\overline{K} = K$.
\end{lemma}
\begin{proof}
Since the closure of a set is the set together with all of its accumulation points, a set equalling its closure means that the set contains all of its limit points. By a \hyperlink{Accumulation Points of a Closed Set}{previous lemma}, this is equivalent to the set being closed.
\end{proof}
\begin{lemma}[Closure of a Set is Closed]
The closure of any set $S$ is a closed set.
\end{lemma}
\begin{proof}
Let $p \in \overline{S}^C$. Since $p$ is not in $S$ nor is it a limit point, there exists a disk around $p$ which is entirely contained in $\overline{S}^C$. The entire complement of the closure is therefore the union of such open disks. But since \hyperlink{Open Disks are Open Sets}{an open disk is an open set} and \hyperlink{Properties of Open Sets}{the union of open sets is open}, the complement of the closure is open, meaning the closure is closed.
\end{proof}
\subsection{Interior, Exterior, and Boundary}
\begin{definition}[Interior of a Set]
Given any set $S$, its \emph{interior} $\interior{S}$ is defined as the complement of the closure of the complement of $S$.
\end{definition}
\begin{lemma}[Properties of the Interior]
If $S$ is some set, and $\interior{S}$ is its interior, the following will hold:
\begin{enumerate}
\item $\interior{S}$ is open.
\item \begin{equation*} \interior{S} = \{p \in S \mid S \in \nbhd{p}\}\end{equation*}
\end{enumerate}
\end{lemma}
\begin{proof}
The interior is an open set because it is the complement of the closure of some other set. Since \hyperlink{Closure of a Set}{the closure of a set is closed}, this means that the interior is open.
To prove the second part, take any point $p$ in the set where the set is a neighborhood of $p$. Clearly, $p$ is not in the complement of $S$. Since there is a neighborhood of $p$ which does not contain any points in the complement of $S$, namely $S$ itself, it is not a limit point of the complement of $S$. Therefore, $p$ is not in the complement of the closure, and therefore is in the complement of the closure of the complement, also called the interior.
\end{proof}
\begin{definition}[Exterior of a Set]
Given any set $S$, its \emph{exterior} $\exterior{S}$ is defined as the interior of its complement.
\end{definition}
\begin{definition}[Boundary of a Set]
Given any set $S$, its \emph{boundary} $\boundary{S}$ consists of all points not in the interior of exterior of $S$.
\end{definition}
\subsection{Sequences in the Complex Plane, Limits of Sequences}
\subsection{Connectedness and Compactness}
\subsection{Limits of Functions, Continuity}
\subsubsection{Continuous Images of Connected and Compact Sets}


\section{COMPLEX DIFFERENTIATION}
\subsection{Differentiability and Analyticity}
\subsubsection{The Cauchy-Riemann Equations}
\subsection{Rules for Derivatives}
\subsection{Conformal Maps}




\section{COMPLEX INTEGRATION}
\subsection{Real Integrals of Complex Functions}
\subsection{Contours and Contour Integration}
\subsubsection{Contour Parameterization}
\subsection{Introduction to Cauchy's Theorem}
\subsubsection{Cauchy's Theorem on Tiles}
\subsubsection{Tile-Centered Regions and the Not-Too-Bad Condition}
\subsection{Cauchy's
 Integral Formula}
\subsubsection{Winding Numbers and Continuous Angle Functions}

\section{TAYLOR SERIES AND SINGULARITIES}
\subsection
{Taylor Series}
\subsection{Meromorphic Functions, Classification of Singularities}
\subsubsection{Properties of Essential Singularities}

\section{HOMOTOPY AND HOMOLOGY}
\subsection{Homotopy}
\subsection{The Homology Group of a Region}
\subsection{Simple Connectivity}

\section{}
\end{document}
